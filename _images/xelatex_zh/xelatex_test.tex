%%xelatex的测试模版,我就没有精简直接发上来,大家参考。
%%其实写这些也参考了很多了网的资料,感谢他们。
%%要pdf版本的可以去我的iask下
%%祝大家学习生活愉快!

\documentclass{article}
\usepackage{xeCJK}  %必须加xeCJK包
\setCJKmainfont{WenQuanYi Micro Hei}  %换成本地字体{Microsoft YaHei}%

\begin{document}

\title{xelatex及中文Gummi在ubuntu上的配置}
\author{Fame}
\maketitle


\section{install xelatex}
reference url:http://bachue.is-programmer.com/posts/26739.html\\
菜鸟在 Ubuntu 11.04 下配置支持中文的LaTeX并生成PDF的简单步骤\\

久闻LaTeX大名,由于Linux下常用的LibreOffice实在太过于糟糕,与Microsoft Office差距太大,并且LibreOffice已经是Linux下最好的Office套件了,使用LaTeX来替代LibreOffice恐怕是一种无奈的选择。\\
但是我从来没有用过LaTeX,连类似的排版工具也从未用过,可以说丝毫没有经验。不过我还是决定试着学习下LaTeX,首先当然是从配置环境开始,我不熟悉LaTeX和类似软件使用的文件格式,可以说,我只认识PDF格式,所以在配置的时候也是奔着PDF而去。网上的LaTeX配置的文章多如牛毛,但是方法竟然都不一样,有简单的,有繁琐的。英文LaTeX环境很容易就可以搞定,但是中文却没有这么简单。我尝试了几个比较简单的方法,全都不可行。复杂的方法,或是说步骤超过5步的,或是要修改配置文件的方法,我全都不尝试,因为一看这种方法就感到头疼。下载TexLive2010也许比较适合我,但是他的大小实在太恐怖,2GB,不知道为什么会这么大,这个大小恐怕已经超过Microsoft Office了。下载的速度超慢,初步算算大约下载2天,不考虑。\\
后来发现,其实最简单的方法就是使用XeLaTeX,好处是:
\begin{enumerate}
\item 无须配置,安装后即可使用
\item 支持中文
\item 直接将文件转为PDF
\end{enumerate}


安装只需要三个包
    
sudo apt-get install texlive-xetex latex-cjk-xcjk texlive-latex-recommended

不过依赖包相当多,apt大约下载半多个小时才能安装。这期间无须任何干涉。

安装完之后,可以先写一个简单的测试文件,当然,要带中文
\begin{quote}

\%\% example.tex
\textbackslash
documentclass{article}\\
\textbackslash
usepackage{xeCJK}  %必须加xeCJK包
\\
\textbackslash
setCJKmainfont{WenQuanYi Micro Hei}  %换成本地字体
\\
\textbackslash {document}\\
chm文件全名Microsoft Compiled HTML Help,看名字也猜得出是微软的格式,Windows操作系统自带它的阅读器,chm普遍用于帮助文件和电子书,但在国内,他的应用范围更广,几乎很多在线文档本地化后都采用这个格式。一般做技术的不可能不接触它,但是一般Linux发行版本不自带chm的阅读器,只能安装软件,chmsee就是我所用过最好的chm阅读器了,他还是一款国产软件,我猜之所以看不到比他更好地国外软件,也许是因为中国人更有需求吧。
 
项目地址:http://code.google.com/p/chmsee/
 
但是Fedora 14的chmsee包有点奇怪,只有1.1.0版本,没有最新的1.3.0版本。1.1.0的Bug很厉害,影响了正常使用,最新版本则修复了这些Bug。但是Fedora 14没有chmsee最新版本,我在网上也没有搜索到适合Fedora的RPM包,惟一的OpenSUSE的RPM包使用yum localinstall安装仍有不能解决的依赖关系,这使得我们不得不用源码来编译安装了。
 
我看到项目网站上解释他没有发布二进制安装文件的原因是“Linux 的发行版太多了,还是只发布源码比较省事”,我不否认这种说法的正确性,但也不能否认源码编译安装对我这种Linux菜鸟而言确实非常困难。主要原因当然是缺乏库文件,而且一旦缺乏,我往往不知道该下载哪个包来解决,如果出现RPM包常见的依赖性地狱,更是让我绝望,所以,除非确实找不到二进制包并且需求这个软件,否则我绝对不会去编译源代码。
 
我看到他的项目文件讲了依赖如下:
cmake >= 2.6.4
gtk2 >= 2.18
xulrunner >= 1.9
chmlib
libxml2
libgcrypt
intltool
幸运的是,Yum库中这些软件全都有,不幸的是,安装了后连cmake都通不过,还是有大量库文件缺乏。后来猜测他的库文件可能不是放在这些可执行的程序中,而是放在devel包中,因此尝试安装了他们的devel包,结果竟然成功了(后来发现项目文件中说的很清楚:Before compiling, you need to check following libraries or their development packages already installed on your system.可惜我一开始没有注意),只要cmake一通过,就不会再有依赖性问题了,安装非常成功。下面复述下正确的依赖。
 
cmake
gtk2-devel
xulrunner-devel
chmlib-devel
libxml2-devel
libgcrypt-devel
intltool
版本方面不用担心,Yum库中的软件都符合要求。
\textbackslash {document}
\end{quote}

这里把文件命名为example.tex。

可以看到,里面动用了一个中文字体——文泉驿微米黑,这是我最喜欢的字体,也是电脑里唯一一个中文字体,你可以修改这个字体来使用你自己喜欢的字体。你可以用
    
fc-list :lang=zh-cn

来输出你电脑上所有中文字体。

编译example.tex非常简单,只需要使用
    
xelatex example.tex

即可获得example.pdf文件,打开即可见到效果。

本人对LaTeX一窍不通,如果你说这样配置出的LaTeX环境会有什么什么缺陷我无法应答,不过我认为这也不要紧。只要有一个简单的工作环境,用它稍微学习下LaTeX,让菜鸟们稍微有一点LaTeX的感觉,和高手们有点共同语言,此时再重新用更好的方法配置LaTeX,才是最人性化的做法。如果一开始就使用那种无法理解的方法配置,想提问又被高手们鄙视,还没开始学习就有了不愉快的感觉,这种做法不利于以后的发展。

另外,求一个LaTeX IDE,简单点就可以了。不太想记很多LaTeX语法。

\section{install latex-cjk}
reference url:http://blog.csdn.net/yangzhuoluo/article/details/5697205\\
Ubuntu中配置LaTeX中文的方法\\

Ubuntu 官方源就带有这个包,只需要 sudo apt-get install latex-cjk-all 即可。

安装好之后,在源文件里添加类似如下的代码,就能正确处理中文了:

 
\begin{quotation}
/documentclass[a4paper,12pt]{article}\\
/usepackage{CJK}\\
/begin{document}\\
/begin{CJK}{UTF8}{gbsn}\\
杨卓荦中文测试\\
/end{CJK}\\
/end{document}\\
\end{quotation}
LateX 中文第一段的首行缩进

用LaTeX时,按照英文的写作风格,第一段是没有首行缩进的,发现用CJK包之后来写中文时,默认的article类的首段的首行也不缩进,这样很不好。

让首行缩进的方法也很简单。加入indentfirst包,然后设置缩进为2个字即可:
/usepackage{indentfirst}\\
/setlength{/parindent}{2em}\\
哪段不想缩进了,加上/noindent 缩进是/indent(默认的)

粗体部分是关键代码,/begin{CJK}后面建议使用符合时代精神的 UTF8 编码而不是 GB,字体使用默认的 gbsn(宋体),这样做就不需要再手工安装任何字体,也不需要再做什么配置。
关于中文文档命名的习惯

中文文档习惯于使用“目录”、“插图目录”、“表格目录”、“参考文献”、“摘要”、“索引”、“表格”、“图”等字样作为文章特殊部分的标题,而 LaTeX 对于这些部分的标题默认是使用英文的,因此可以通过重定义宏的方式将其重定义为中文字样。即加入如下代码:

\% 中文文档习惯 /renewcommand{/contentsname}{目录}\\
/renewcommand{/listfigurename} {插图目录}\\
/renewcommand{/listtablename} {表格目录}\\
/renewcommand{/refname}{参考文献}\\
/renewcommand{/abstractname}{摘要}\\
/renewcommand{/indexname}{索引}\\
/renewcommand{/tablename}{表}\\
/renewcommand{/figurename}{图}\\

\section{install Gummi}
reference url1:http://www.linuxeden.com/html/news/20100926/104890.html\\
Gummi: 简易 LaTeX 编辑工具\\
\\
reference url2:http://www.charlietanksley.net/philtex/forum/topic/links-to-the-most-popular-latex-editors\\
Links to the most popular LaTeX editors\\

Gummi 是一个利用 Python 和 GTK 技术开发的简易 LaTeX 编辑工具,具备实时预览、语法高亮、BibTex 支持、导出到 PDF 、纠错/拼写检查、预设模版等功能。
Gummi: 简易 LaTeX 编辑工具
项目主页: http://gummi.midnightcoding.org/

\# Ubuntu 用户可以通过这个 PPA 来安装:
    sudo apt-add-repository ppa:gummi/gummi\\
    sudo apt-get update\\
    sudo apt-get install gummi\\
\\
Here is a comparison(only top ones are listed here):
Classic Editors:\\
AUCTeX extension for the classic open source GNU Emacs text editor\\
    GeanyLaTeX: LaTeX plugin for the lightweight GTK programming editor, Geany\\
    Gedit LaTeX plugin: Extension for the Gedit text editor of the GNOME free desktop environment\\
    Gummi: Simple open source Linux/GTK LaTeX editor with live preview\\
    KILE: Open source LaTeX IDE for KDE (the K desktop environment)\\
    LEd: Freeware LaTeX editor for MS Windows\\
    LyX: Crossplatform, open souce WYSIWYM (What-You-See-Is-What-You-Mean) editor that produces TeX output\\
    MeWa: Open source LaTeX editor for MS Windows\\
    Notepad+: Powerful and extensible general purpose open source text editor for MS Windows\\
    
\end{document}
